\subsubsection{Funcionamiento}

El funcionamiento del paquete es muy sencillo, básicamente, la idea es crear una o varias ventanas con una configuración deseada y acto seguido ir creando y añadiendo secciones (componentes), donde posteriormente se pueden ir borrando y actualizando los componentes individual o grupalmente. Por ello, este paquete proporciona una serie de métodos para la ventana y las secciones.\bigskip

\noindent\textbf{¿Cómo funciona la ventana?}\bigskip

Para hacer que la ventana funcione es necesario crear una instancia del objeto \texttt{Window}, inicializar la configuración de la misma empleando el método \texttt{initConfig} recibiendo por argumento la ruta donde se encuentra el archivo. En caso contrario se empleará la predeterminada. Finalmente, para que se visualice la ventana se realizará con el método \texttt{start} que visualizará una ventana con la configuración indicada anteriormente.\bigskip

En el caso de querer añadir componentes como una barra de progreso, un botón o derivados, es tan sencillo como crear una sección y emplear el método \texttt{addSection}, el cual recibe por parámetro la sección a añadir, el nombre de la sección y la posición de este en la ventana.\bigskip

\noindent\textbf{¿Cómo funciona la sección?}\bigskip

Las secciones son instancias de la clase \texttt{Section} que permiten formar los diferentes componentes de la ventana.\bigskip

El funcionamiento de una sección es muy sencilla, se trata de instanciar un objeto de la clase \texttt{Section} y llamar a algún método de esta clase, pasando los parámetros adecuados, y finalmente añadir la sección a la ventana.\bigskip

\noindent\textbf{Otros}\bigskip

Adicionalmente, se dispone de una clase llamada \texttt{DirectionAndPosition}, que constituye las posibles orientaciones y direcciones que una sección puede tener.
