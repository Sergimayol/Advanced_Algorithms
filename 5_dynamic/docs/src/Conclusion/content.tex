\section{Conclusión}

Teniendo en cuenta el desarrollo expuesto, la aplicación permite generar un punto de referencia de un conjunto de algoritmos de diferente complejidad asintótica mediante una representación gráfica e interactiva de la media de los tiempos de ejecución a través de unos parámetros de entrada definidos por el usuario. Adicionalmente, se ha implementado mediante una versión modificada del patrón de diseño de software \textbf{M}odelo \textbf{V}ista \textbf{C}ontrolador (MVC) añadiendo un cuatro módulos que permite la comunicación entre los diferentes elementos de este mediante peticiones. Esta modificación ha sido fuertemente inspirada en el diseño de un servidor, lo que permite gran flexibilidad y escalabilidad de desarrollo, al poder interceptar y gestionar las peticiones a gusto del desarrollador.\bigskip

Además, se ha hecho énfasis en el uso de las librerías creadas por los miembros del grupo para facilitar el trabajo y la reutilización de código en futuras prácticas. Reiterar que el principal motivo que nos ha impulsado a implementar dichas librerías es la facilidad que proporcionan para centrarse únicamente en el desarrollo de la práctica en sí; Gracias a que durante el desarrollo de estas librerías se ha priorizado la facilidad de manejo, uso genérico y optimización para añadir el mínimo \say{overhead} al rendimiento del programa.\bigskip

Concluir que, con respecto al ámbito académico, este proyecto nos ha permitido consolidar el concepto de patrón de diseño MVC gracias a un primer proceso de investigación y discusión del diseño a implementar entre los integrantes del grupo.