\subsubsection{Modelo}

El modelo es la representación de los datos que maneja el software. Contiene los mecanismos y la lógica necesaria para acceder a la información y para actualizar el estado del modelo.\bigskip

Esta clase contiene diferentes tipos de estructuras de datos específicas a esta práctica, incluyendo, pero no solamente, la iteración actual del programa, el tamaño del lote, los tiempos de ejecución de todos los algoritmos implementados y el \say{timeout}.\bigskip

Adicionalmente de los datos mencionados previamente, provee unos métodos o funciones auxiliares que facilitan la modificación o adquisición de los datos:\bigskip

\texttt{resetData} y \texttt{resetIterations} permiten poner los valores por defecto de los principales datos del modelo.\bigskip

\texttt{nextIteration} actualiza el tamaño del vector de datos para la siguiente iteración y aumenta el contador de iteraciones.\bigskip

\texttt{collectData} obtiene los resultados de la última iteración presente en el controlador de manera \say{inteligente}, al tratar los datos de diferente manera dependiendo del estado del controlador.\bigskip

\texttt{getData} devuelve todos los tiempos de ejecución hasta ese instante convertidos mediante la función de representación seleccionada en la vista.\bigskip

Finalmente, al ser un módulo de nuestro MVC, implementa la interfaz \texttt{Notify} y su método \texttt{notifyRequest} que le permite recibir notificaciones de los otros módulos del MVC.
