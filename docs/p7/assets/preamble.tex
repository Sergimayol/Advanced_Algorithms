\usepackage{babel}
\usepackage{dirtytalk}
\usepackage{enumitem}
\usepackage{minted}
\usepackage[table,xcdraw,usenames,dvipsnames]{xcolor}   % Colored text

% Background color for minted and text coloring style
\definecolor{codebg}{rgb}{0.95,0.95,0.95}
\usemintedstyle{friendly}

% Code enviroment. Escape with "@"
\newenvironment{code}[2]
 {\VerbatimEnvironment
  \begin{minted}[fontsize=#1, linenos,breaklines,numbersep=8pt,gobble=0,frame=lines,bgcolor=codebg,framesep=3mm,escapeinside=@@]{#2}}
 {\end{minted}}

 
% Palatino/Palladio as is done in IEEE Computer Society journals.
% To go back to Times Roman, you can use this code:
%\renewcommand{\rmdefault}{ptm}\selectfont

% *** CITATION PACKAGES ***
%
\ifCLASSOPTIONcompsoc
  % The IEEE Computer Society needs nocompress option
  % requires cite.sty v4.0 or later (November 2003)
  \usepackage[nocompress]{cite}
\else
  % normal IEEE
  \usepackage{cite}
\fi


% *** MATH PACKAGES ***
%
\usepackage{amsmath}
% A popular package from the American Mathematical Society that provides
% many useful and powerful commands for dealing with mathematics.
%
% Note that the amsmath package sets \interdisplaylinepenalty to 10000
% thus preventing page breaks from occurring within multiline equations. Use:
%\interdisplaylinepenalty=2500
% after loading amsmath to restore such page breaks as IEEEtran.cls normally
% does. amsmath.sty is already installed on most LaTeX systems. The latest
% version and documentation can be obtained at:
% http://www.ctan.org/pkg/amsmath


% *** SPECIALIZED LIST PACKAGES ***

\usepackage{algorithmic}
% algorithmic.sty was written by Peter Williams and Rogerio Brito.
% This package provides an algorithmic environment fo describing algorithms.
% You can use the algorithmic environment in-text or within a figure
% environment to provide for a floating algorithm. Do NOT use the algorithm
% floating environment provided by algorithm.sty (by the same authors) or
% algorithm2e.sty (by Christophe Fiorio) as the IEEE does not use dedicated
% algorithm float types and packages that provide these will not provide
% correct IEEE style captions. The latest version and documentation of
% algorithmic.sty can be obtained at:
% http://www.ctan.org/pkg/algorithms
% Also of interest may be the (relatively newer and more customizable)
% algorithmicx.sty package by Szasz Janos:
% http://www.ctan.org/pkg/algorithmicx


% *** ALIGNMENT PACKAGES ***
%
\usepackage{array}
% Frank Mittelbach's and David Carlisle's array.sty patches and improves
% the standard LaTeX2e array and tabular environments to provide better
% appearance and additional user controls. As the default LaTeX2e table
% generation code is lacking to the point of almost being broken with
% respect to the quality of the end results, all users are strongly
% advised to use an enhanced (at the very least that provided by array.sty)
% set of table tools. array.sty is already installed on most systems. The
% latest version and documentation can be obtained at:
% http://www.ctan.org/pkg/array


\usepackage{eqparbox}
% Also of notable interest is Scott Pakin's eqparbox package for creating
% (automatically sized) equal width boxes - aka "natural width parboxes".
% Available at:
% http://www.ctan.org/pkg/eqparbox


% *** PDF, URL AND HYPERLINK PACKAGES ***

% NOTE: PDF hyperlink and bookmark features are not required in IEEE
%       papers and their use requires extra complexity and work.
% *** IF USING HYPERREF BE SURE AND CHANGE THE EXAMPLE PDF ***
% *** TITLE/SUBJECT/AUTHOR/KEYWORDS INFO BELOW!!           ***
\newcommand\MYhyperrefoptions{bookmarks=true,bookmarksnumbered=true,
pdfpagemode={UseOutlines},plainpages=false,pdfpagelabels=true,
colorlinks=true,linkcolor={black},citecolor={black},urlcolor={black},
pdftitle={Bare Demo of IEEEtran.cls for Computer Society Journals},%<!CHANGE!
pdfsubject={Typesetting},%<!CHANGE!
pdfauthor={Michael D. Shell},%<!CHANGE!
pdfkeywords={Computer Society, IEEEtran, journal, LaTeX, paper,
             template}}%<^!CHANGE!
\ifCLASSINFOpdf
\usepackage[\MYhyperrefoptions,pdftex]{hyperref}
\else
\usepackage[\MYhyperrefoptions,breaklinks=true,dvips]{hyperref}
\usepackage{breakurl}
\fi
% One significant drawback of using hyperref under DVI output is that the
% LaTeX compiler cannot break URLs across lines or pages as can be done
% under pdfLaTeX's PDF output via the hyperref pdftex driver. This is
% probably the single most important capability distinction between the
% DVI and PDF output. Perhaps surprisingly, all the other PDF features
% (PDF bookmarks, thumbnails, etc.) can be preserved in
% .tex->.dvi->.ps->.pdf workflow if the respective packages/scripts are
% loaded/invoked with the correct driver options (dvips, etc.). 
% As most IEEE papers use URLs sparingly (mainly in the references), this
% may not be as big an issue as with other publications.
%
% That said, Vilar Camara Neto created his breakurl.sty package which
% permits hyperref to easily break URLs even in dvi mode.
% Note that breakurl, unlike most other packages, must be loaded
% AFTER hyperref. The latest version of breakurl and its documentation can
% be obtained at:
% http://www.ctan.org/pkg/breakurl
% breakurl.sty is not for use under pdflatex pdf mode.
%
% The advanced features offer by hyperref.sty are not required for IEEE
% submission, so users should weigh these features against the added
% complexity of use.
% The package options above demonstrate how to enable PDF bookmarks
% (a type of table of contents viewable in Acrobat Reader) as well as
% PDF document information (title, subject, author and keywords) that is
% viewable in Acrobat reader's Document_Properties menu. PDF document
% information is also used extensively to automate the cataloging of PDF
% documents. The above set of options ensures that hyperlinks will not be
% colored in the text and thus will not be visible in the printed page,
% but will be active on "mouse over". USING COLORS OR OTHER HIGHLIGHTING
% OF HYPERLINKS CAN RESULT IN DOCUMENT REJECTION BY THE IEEE, especially if
% these appear on the "printed" page. IF IN DOUBT, ASK THE RELEVANT
% SUBMISSION EDITOR. You may need to add the option hypertexnames=false if
% you used duplicate equation numbers, etc., but this should not be needed
% in normal IEEE work.
% The latest version of hyperref and its documentation can be obtained at:
% http://www.ctan.org/pkg/hyperref

% correct bad hyphenation here
\hyphenation{op-tical net-works semi-conduc-tor}
\usepackage{csquotes}
