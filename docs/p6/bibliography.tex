\begin{thebibliography}{1}

\bibitem{betterswing}
Better Swing, \emph{An easy way to develop java GUI apps}, V0.0.3. \hskip 1em plus
  0.5em minus 0.4em\relax Ver documentación completa \href{https://sergimayol.github.io/better-swing}{\textcolor{blue}{aquí}}.

\bibitem{Graphical engine}
Graphical engine, \emph{How to develop a graphical engine from scratch}. Ver enlace \href{https://www.youtube.com/watch?v=025QFeZfeyM&t=7334s}{\textcolor{blue}{aquí}}.

\bibitem{Gson}
Google json (Gson), \emph{Java serialization/deserialization library to convert Java Objects into JSON and back}. Ver enlace \href{https://github.com/google/gson}{\textcolor{blue}{aquí}}.

\bibitem{NodeJs architecture}
Node.js Architecture, \emph{Architecture of a single thread cross-platform, open-source server environment}. Ver enlace \href{https://codedamn.com/news/nodejs/node-js-architecture}{\textcolor{blue}{aquí}}.

\bibitem{Sqlite database}
The SQLite database, a serverless database. \emph{Quick start and documentation}. Ver enlace \href{https://www.sqlite.org/index.html}{\textcolor{blue}{aquí}}.

\bibitem{JFreeChart}
Open source Java chart library. \emph{An easy way to create and visualize charts in Java}. Ver enlace \href{https://www.jfree.org/jfreechart/}{\textcolor{blue}{aquí}}.

\bibitem{Method Overloading}
A way to define different behavior for a method with the same name. Ver enlace \href{https://www.geeksforgeeks.org/method-overloading-in-java/}{\textcolor{blue}{aquí}}

\bibitem{Puzzle 15}
What is the puzzle 15 game? How does it work?. Ver enlace \href{https://es.wikipedia.org/wiki/Juego_del_15}{\textcolor{blue}{aquí}}

\bibitem{Branch and Bound}
Branch and bound concept and algorithm. Ver enlace \href{https://en.wikipedia.org/wiki/Branch_and_bound}{\textcolor{blue}{aquí}}

\end{thebibliography}