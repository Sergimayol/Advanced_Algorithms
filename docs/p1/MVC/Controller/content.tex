\subsubsection{Controlador}
El controlador en un MVC es el responsable de recibir y procesar la entrada del usuario y actualizar el modelo. En esta práctica, dicha responsabilidad es la de calcular el tiempo de ejecución de los siguientes algoritmos:\medskip

\begin{itemize}[leftmargin=0pt]
    \item[] La moda de un vector numérico en $O(n)$
    \item[] La moda de un vector numérico en $O(n log n)$
    \item[] El producto escalar entre dos vectores numéricos en $O(n^2)$
\end{itemize}\medskip

Para la implementación de estos algoritmos se ha decidido seguir una convención de programación declarativa, aunque se ha implementado la versión imperativa en el caso de que el desarrollador los prefiera usar. Adicionalmente, el propio controlador produce los vectores de datos usados como argumentos en los algoritmos mediante un generador aleatorio, haciendo su ejecución lo más transparente posible.\bigskip

La ejecución del cálculo se realiza en un thread virtual, obteniendo una mejora en el tiempo de respuesta de la aplicación; permitiendo que otros eventos puedan tomar el thread principal, aprovechando los cores de la CPU y reduciendo el bloqueo de la aplicación; permitiendo que el thread principal sea menos probable de quedarse bloqueado debido a cálculos de larga duración.\bigskip

Una vez acabado los cálculos, crea una petición al hub para que el modelo pueda obtener los resultados. Esto es posible debido a que, al ser un módulo del MVC modificado, implementa la interfaz \texttt{Notify} y el método \texttt{notifyRequest} que le permite comunicarse con los otros módulos. 
