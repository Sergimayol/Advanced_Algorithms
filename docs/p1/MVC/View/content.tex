\subsubsection{Vista}

La vista contiene los componentes para representar la interfaz del usuario (IU) del programa y las herramientas con las cuales el usuario puede interactuar con los datos de la aplicación. Adicionalmente, la vista se encarga de recibir e interpretar adecuadamente los datos obtenidos del modelo.\bigskip

\texttt{loadContent} es el encargado de cargar el contenido en la ventana. Para esta práctica, carga los siguientes elementos semánticos:\bigskip

\texttt{header} del programa mediante \texttt{createButtons}, generando el conjunto de botones que permiten al usuario del programa lanzar cada uno de los algoritmos e incluso pausar su ejecución en cualquier momento. Para ser más específicos, los botones generados son: \textit{Escalar}, \textit{Moda n Log n}, \textit{Moda n}, \textit{Todos} y \textit{Reanudar/Pausar}.\bigskip

\texttt{main} del programa mediante \texttt{updateChart}, cargando el gráfico con respecto a las variables por defecto encontradas en el modelo.\bigskip

\texttt{footer} del programa que ocupará la zona inferior de la interfaz, conteniendo, las barras de progreso para alcanzar el tiempo límite de cada algoritmo, el tamaño del lote, la representación temporal, además de la iteración actual y la ponderación por iteración.\bigskip

Adicionalmente, la clase contiene un conjunto de métodos o funciones que facilitan la generación de los elementos en la interfaz de usuario. A continuación se explicarán brevemente los más importantes:\bigskip

\texttt{createProgressBarToTimeOut} crea una barra de progreso que, para esta práctica, estará asociada al progreso con respecto al timeout especificado en el modelo de un algoritmo. Siendo la barra del algoritmo Escalar de color rojo, el algoritmo ModeNLogN de color azul y el algoritmo Moda N de color verde, siendo así consistentes con la leyenda de la gráfica.\bigskip

\texttt{createButtons} crea todos los botones que aparecen en el header de la interfaz, es decir, los botones: \textit{Escalar}, \textit{Moda n Log n}, \textit{Moda n}, \textit{Todos} y \textit{Pausar/Reanudar}. Adicionalmente, por cada botón se le asigna un \say{listener} que generará la petición correspondiente a la etiqueta del botón.\bigskip

\texttt{footer} crea un menú en la parte inferior de la interfaz con las siguientes opciones:

\begin{itemize}
    \item Una opción para seleccionar la representación del tiempo permitiendo seleccionar entre Nanosegundos, Milisegundos, Segundos, Minutos, Horas o Días.
    \item Una opción para seleccionar el tamaño del lote. Cuando se cambia el tamaño del lote se reinician todos los datos.
    \item Una opción para seleccionar el número de iteraciones. Cuando se cambia el número de iteraciones se reinician todos los datos.
\end{itemize}\bigskip
