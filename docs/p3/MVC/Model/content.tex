\subsection{Modelo}

El modelo es la representación de los datos que maneja el software. Contiene los mecanismos y la lógica necesaria para acceder a la información y para actualizar el estado del modelo.\bigskip

Esta clase contiene diferentes tipos de estructuras de datos específicas a esta práctica, incluyendo, pero no solamente: \\

\begin{description}
\item[hub] Permite la comunicación con el controlador y la vista.
\item[seed] Representa el número de la semilla.
\item[pointAmount] Representa el número de puntos.
\item[data] Array de puntos para guardar los datos. 
\item[solutionsForMaxNN] Array de soluciones para el algoritmo N*N buscando la distancia máxima.
\item[solutionsForMinNN] Array de soluciones para el algoritmo N*N buscando la distancia mínima.
\item[solutionsForMinNLogN] Array de soluciones para el algoritmo N*logN buscando la distancia mínima.
\item[nSolutions] Número de soluciones a representar
\item[useNLogNAlgorithm] Booleano para seleccionar el tipo de algoritmo a aplicar.
\item[useMaxOnAuto] Booleano para iterar el algoritmo en caso de que sea true.
\item[Lambda] Almacena el valor de lambda. 
\end{description}

Finalmente, al ser un módulo de nuestro MVC, implementa la interfaz \texttt{Notify} y su método \texttt{notifyRequest} que le permite recibir notificaciones de los otros módulos del MVC.

\subsubsection{Estructuras de datos}
Para esta práctica se han generado tres estructuras de datos de lectura para facilitar la encapsulación de información. Para implementarlas se ha aprovechado de la estructura \say{record} de Java. A continuación se mencionan y explican las estructuras:

\begin{description}
\item[Point] Representa un punto en la nube de datos.
\item[PairPoint] Permite guardar una pareja de \say{Point's}.
\item[Solution] Permite guardar un \say{PairPoint} junto a su distancia y su tiempo para encontrarlos como solución, es decir, esta estructura, básicamente, contiene una solución del algoritmo.
\end{description}
